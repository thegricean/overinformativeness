\documentclass[11pt]{article}

\usepackage{apacite}
\usepackage{amsmath,amssymb}
\usepackage{graphicx}
\usepackage{color}
\usepackage{url}
\usepackage{fullpage}
\usepackage{setspace}
\usepackage{booktabs}
\usepackage{lingmacros}

%\newcommand{\url}[1]{$#1$}

\definecolor{Blue}{RGB}{50,50,200}
\newcommand{\blue}[1]{\textcolor{Blue}{#1}}

\definecolor{Red}{RGB}{255,0,0}
\newcommand{\red}[1]{\textcolor{Red}{#1}}
\newcommand{\jd}[1]{\textcolor{Red}{[jd: #1]}} 

\definecolor{Green}{RGB}{50,200,50}
\newcommand{\ndg}[1]{\textcolor{Green}{[ndg: #1]}}  

 \newcommand{\denote}[1]{\mbox{ $[\![ #1 ]\!]$}}


\newcommand{\subsubsubsection}[1]{{\em #1}}
\newcommand{\eref}[1]{(\ref{#1})}
\newcommand{\tableref}[1]{Table \ref{#1}}
\newcommand{\figref}[1]{Figure \ref{#1}}
\newcommand{\appref}[1]{Appendix \ref{#1}}
\newcommand{\sectionref}[1]{Section \ref{#1}}


\title{Rethinking overinformativeness: redundant use of modifiers in referring expressions}

 
\author{{\large \bf Judith Degen, Noah D.~Goodman} \\
  \{jdegen,ngoodman\}@stanford.edu\\
  Department of Psychology, 450 Serra Mall \\
  Stanford, CA 94305 USA}

\begin{document}

\maketitle


\begin{abstract}
 

\textbf{Keywords:} 
reference; referring expressions; informativeness; probabilistic pragmatics; experimental pragmatics
\end{abstract}

\section{Introduction}
\label{sec:intro}

Referring is one of the most basic and prevalent uses of language, so we're probably pretty good at it. What does `being good at it' mean? Generally, trying to abide by Gricean maxims, including: be as informative as possible but not more informative than necessary. For the past 40 years, researchers have been noticing that speakers don't seem to abide by the second part of that in their production of referring expressions: they often include modifiers that aren't necessary for uniquely determining reference. This has posed a challenge for rational accounts of language use (including the Gricean one): what accounts for this extra expenditure of useless effort on the speaker's part? Is it just that speakers aren't economical after all? Or is there some utility in being redundant?

%\begin{itemize}
	%\item 
	\red{introduce the general issue: content selection, grice, the term `overinformativeness'}
%\end{itemize}

\subsection{Asymmetry in redundant use of color and size adjectives}
\label{sec:asymmetry}

\subsection{Number of distractors}
\label{sec:numdistractors}

\subsection{Scene variation}
\label{sec:scenevariation}

\subsection{Color typicality}
\label{sec:colortypicality}

\subsection{Summary}
\label{sec:introsummary}

\begin{table}
\begin{tabular}{l p{6cm} p{5.5cm} }
\toprule
Effect & Description & Reported by \\
\midrule
Color/size asymmetry & More redundant use of color adjectives than size adjectives &  \citeA{Pechmann1989, engelhardt2006a, gatt2011} \red{others}\\
Number of distractors & More redundant use of color with increasing number of distractors & ?? \red{deutsch} \\
Scene variation & More redundant use of color with increasing scene variation & \citeA{davies2009, koolen2013}\\
Color typicality & More redundant use of color with decreasing color typicality & \citeA{sedivy2003a, Westerbeek2014}\\
\bottomrule
\end{tabular}
\end{table}

\red{insert table here}

\section{Modeling speakers' choice of referring expression}
\label{sec:models}

\red{introduce the basic model and show why it doesn't get overinformative}

\subsection{Extending the basic model with noisy truth functions}
\label{sec:noisytruthmodel}

\red{show the model extension and the predictions it makes for the basic color/size asymmetry. report our replication of gatt?}

\subsection{Number of distractors}
\label{sec:numdistmodel}

\red{show the model's predictions for varying number of distractors. show how those predictions vary as a function of the type of distractors -- novel! report experiment! (to be done)}

\subsection{Scene variation}
\label{sec:scenevarmodel}

\red{show model predictions for koolen et al scenes -- qualitative effect}

\subsection{Color typicality}
\label{sec:colortypmodel}

\red{set up model extension to get color typicality effect (or will you end up reporting this model from the start?)}

\section{General Discussion}
\label{sec:gd}



\appendix

\bibliographystyle{apacite}

\setlength{\bibleftmargin}{.125in}
\setlength{\bibindent}{-\bibleftmargin}

\bibliography{bibs}


\end{document}
