\documentclass[11pt]{article}

\usepackage{apacite}
\usepackage{amsmath,amssymb}
\usepackage{graphicx}
\usepackage{color}
\usepackage{url}
\usepackage{fullpage}
\usepackage{setspace}
\usepackage{booktabs}
\usepackage{lingmacros}
\usepackage{caption}
\usepackage{subcaption}

%\newcommand{\url}[1]{$#1$}

\definecolor{Blue}{RGB}{50,50,200}
\newcommand{\blue}[1]{\textcolor{Blue}{#1}}

\definecolor{Red}{RGB}{255,0,0}
\newcommand{\red}[1]{\textcolor{Red}{#1}}
\newcommand{\jd}[1]{\textcolor{Red}{[jd: #1]}} 

\definecolor{Green}{RGB}{50,200,50}
\newcommand{\ndg}[1]{\textcolor{Green}{[ndg: #1]}}  

 \newcommand{\denote}[1]{\mbox{ $[\![ #1 ]\!]$}}


\newcommand{\subsubsubsection}[1]{{\em #1}}
\newcommand{\eref}[1]{(\ref{#1})}
\newcommand{\tableref}[1]{Table \ref{#1}}
\newcommand{\figref}[1]{Figure \ref{#1}}
\newcommand{\appref}[1]{Appendix \ref{#1}}
\newcommand{\sectionref}[1]{Section \ref{#1}}


\title{Rethinking overinformativeness: redundant use of modifiers in referring expressions}

 
\author{{\large \bf Judith Degen, Noah D.~Goodman} \\
  \{jdegen,ngoodman\}@stanford.edu\\
  Department of Psychology, 450 Serra Mall \\
  Stanford, CA 94305 USA}

\begin{document}

\maketitle


\begin{abstract}
 

\textbf{Keywords:} 
reference; referring expressions; informativeness; probabilistic pragmatics; experimental pragmatics
\end{abstract}

\section{Introduction}
\label{sec:intro}

Referring is one of the most basic and prevalent uses of language, so we're probably pretty good at it. What does `being good at it' mean? Generally, trying to abide by Gricean maxims, including: be as informative as possible but not more informative than necessary. For the past 40 years, researchers have been noticing that speakers don't seem to abide by the second part of that in their production of referring expressions: they often include modifiers that aren't necessary for uniquely determining reference. This has posed a challenge for rational accounts of language use (including the Gricean one): what accounts for this extra expenditure of useless effort on the speaker's part? Is it just that speakers aren't economical after all? Or is there some utility in being redundant?

%\begin{itemize}
	%\item 
	\red{introduce the general issue: content selection, grice, the term `overinformativeness'}
%\end{itemize}

\subsection{Asymmetry in redundant use of color and size adjectives}
\label{sec:asymmetry}

\subsection{Number of distractors}
\label{sec:numdistractors}

\subsection{Scene variation}
\label{sec:scenevariation}

\subsection{Color typicality}
\label{sec:colortypicality}

\subsection{Summary}
\label{sec:introsummary}

\begin{table}
\begin{tabular}{l p{6cm} p{5.5cm} }
\toprule
Effect & Description & Reported by \\
\midrule
Color/size asymmetry & More redundant use of color adjectives than size adjectives &  \citeA{Pechmann1989, engelhardt2006a, gatt2011} \red{others}\\
Number of distractors & More redundant use of color with increasing number of distractors & ?? \red{deutsch} \\
Scene variation & More redundant use of color with increasing scene variation & \citeA{davies2009, koolen2013}\\
Color typicality & More redundant use of color with decreasing color typicality & \citeA{sedivy2003a, Westerbeek2014}\\
\bottomrule
\end{tabular}
\end{table}

\red{insert table here}

\section{Modeling speakers' choice of referring expression}
\label{sec:models}

\red{introduce the basic model and show why it doesn't get overinformative}

\subsection{Extending the basic model with noisy truth functions}
\label{sec:noisytruthmodel}

\red{show the model extension and the predictions it makes for the basic color/size asymmetry. report our replication of gatt?}

\subsection{Number of distractors}
\label{sec:numdistmodel}

The extended noisy semantics model makes a number of novel predictions for the rate of redundant adjective use. In particular, we can ask what the predicted rate of redundant adjective use is as a function of a) the number of distractor items and b) the features of those distractor objects. It has been shown in the past that overmodification increases with the number of distractors that are present \cite{bla}. It has also been established that there is more overmodification in polychrome rather than monochrome displays \cite{bla -- see paula's paper} and in scenes in which objects differ along more feature dimensions \cite{koolen bla}. \red{or save the scene variation stuff for the next section?}

The noisy semantics model's predictions are shown in \figref{fig:numdistractors}. \red{spell out the different conditions and why they're interesting -- why not just number of distractors, why TYPE as well?}

\begin{figure}
\label{fig:numdistractors}
\end{figure}

To test the model's predictions, we conducted two interactive web-based written production studies within a reference game setting.\footnote{See \appref{app:replication}  for a validation of the general paradigm, in which we qualitatively replicate the findings of \citeA{gatt2011} with a different set of stimuli.} Speakers and listeners were shown arrays of objects of that could vary in color and size. Speakers were to produce a referring expression to allow the listener to identify a target object. We manipulated the number of distractor objects in the grid, as well as the variation in color and size among distractor objects.

\subsubsection{Method}

\paragraph{Participants}

We recruited 60 pairs of participants (120 participants total) over Amazon's Mechanical Turk who were each paid \$1.75 for their participation. Here and in all other experiments reported in this paper, participants� IP address was limited to US addresses only and only participants with a past work approval rate of at least 95\% were accepted. The data of five pairs were excluded from analysis because one of the participants dropped out before completing the experiment.

\paragraph{Procedure}

Participants were paired up via \red{is there something in robert's tools to cite?}. For each pair, one participant was assigned the speaker role and one the listener role. They  initially received written instructions \red{more info}. Before continuing to the experiment, participants were required to correctly answer a series of questions about the experimental procedure. These questions are listed in \appref{app:numdistractors}

On each trial participants saw an array of objects. The array contained the same objects for both speaker and listener, but the order of objects was randomized and was typically different for speaker and listener. In the speaker's display, one of the objects -- henceforth the \emph{target} -- was highlighted with a green border. See \figref{fig:exampledisplay} for an example of the listener's and speaker's view on a particular trial.

\begin{figure}
\label{fig:exampledisplay}
\red{insert example display}
\end{figure}

The speaker produced a referring expression to communicate the target to the listener by typing in a chat window. After pressing Enter or clicking the `Send' button, the speaker's message was shown to the listener. The listener then clicked on the object they thought was the target, given the speaker's message.  Once the listener clicked an object, a red border appeared around that object in both the listener and the speaker's display for 1 second before advancing to the next trial.

Both speakers and listeners could write in the chat window, allowing listeners to request clarification if necessary. Listeners could only click on an object and advance to the next trial once the speaker had sent a message. 

\begin{figure}
\begin{subfigure}{\textwidth}
\includegraphics[width=\textwidth]{pics/speaker-perspective-small.png}
\caption{Speakers' perspective}
\label{fig:speakerpersp}
\end{subfigure}

\begin{subfigure}{\textwidth}
\includegraphics[width=\textwidth]{pics/listener-perspective-small.png}
\caption{Listeners' perspective.}
\label{fig:listenerpersp}
\end{subfigure}
\caption{Example displays from the  (a) speaker's and the  (b)  listener's perspective on a \emph{size-sufficient 4-2} trial.}
\label{fig:speakerlistenerperspective}
\end{figure}

\paragraph{Materials}

Participants proceeded through 72 trials. Of these, half were critical trials of interest and half were filler trials. On critical trials, we varied the feature that was sufficient to mention for uniquely establishing reference, the total number of objects in the array, and  the number of objects that shared the non-sufficient feature with the target. 

Objects varied in color and size. On 18 trials, color was a sufficient property for distinguishing the target. On the other 18 trials, size was sufficient. See \figref{fig:speakerlistenerperspective} for an example of a size-sufficient trial from both the speaker's and the listener's perspective. 


In Exp.~1, the number of distractor objects in each array was either 2 (12 trials) or 4 (24 trials). In Exp.~2, the number of distractor objects in each array was either 2, 3, or 4. Finally, we varied the number of distractors that shared the non-sufficient property with the target. That is, when size was the sufficiently distinguishing property, we varied the number of distractors that shared the same color as the target. This number had to be at least one, since otherwise the non-sufficient property would have been sufficient for uniquely establishing reference, i.e.~it would not have been redundant to mention it. 

In Exp.~1, each total number of distractors (2, 4) was crossed with each possible number of distractors that shared the non-sufficient property, leading to the following six conditions: \emph{2-1, 2-2, 4-1, 4-2, 4-3,} and \emph{4-4}, where the first number indicates the total number and the second number the shared number of distractors. Exp.~2 was conducted in order to additionally collect data for the \emph{3-X} conditions. To this end, Exp.~2 included the \emph{3-1, 3-2}, and \emph{3-3} conditions, as well as three conditions repeated from Exp.~1, in order to ensure that results were comparable. The repeated conditions were \emph{2-1, 4-1}, and \emph{4-3}. An overview of the number of trials in each condition and experiment is provided in \tableref{tab:conditions}.

\begin{table}
\caption{Number of trials in each condition.}
\centering
	\begin{tabular}{c c c c}
	\toprule
	Distractors & Feature-sharing distractors & Trials & Exp.\\
	\midrule
	2 & 1 & 12 & 1 \& 2\\
	2 & 2 & 6 & 1\\
	\midrule
	3 & 1 & 6 & 2\\
	3 & 2 & 6 & 2\\
	3 & 3 & 6 & 2\\
	\midrule
	4 & 1 & 12 & 1 \& 2\\
	4 & 2 & 6 & 1\\
	4 & 3 & 12 & 1 \& 2\\
	4 & 4 & 6 & 1\\			
	\bottomrule
	\end{tabular}
\label{tab:conditions}	
\end{table}

Fillers were target trials from a different experiment \cite{graf-underreview}. Each filler item contained a three-object grid. None of the filler objects occurred on target trials. Objects stood in various taxonomic relations to each other and required neither size nor color mention for unique reference. For example, a filler trial could consist of a display with a dalmatian, a pug, and a squirrel.



\subsubsection{Results}

\begin{figure}
\centering
\includegraphics[width=\textwidth]{pics/model-empirical}
\caption{Model predicted probability of redundancy (first column) alongside empirical proportions of redundant descriptions (second column) as a function of the ratio of the number of distractors that differ from the target along the non-sufficient dimension to the number of distractors that do not. Rows indicate the redundant feature. \red{Model params: fcolor .999, fsize .8, lambda 15, costs .1}}
\label{fig:modelexpresults}
\end{figure}

\subsection{Scene variation}
\label{sec:scenevarmodel}

\red{show model predictions for koolen et al scenes -- qualitative effect}

\subsection{Color typicality}
\label{sec:colortypmodel}

\red{set up model extension to get color typicality effect (or will you end up reporting this model from the start?)}

\section{General Discussion}
\label{sec:gd}



\appendix

\section{Validation of interactive web-based written production paradigm}
\label{app:replication}

\red{make sure to discuss why overall we have lower overspecification rates -- probably because of color typicality!! we had pretty typical colors in our stimuli}

\section{Pre-experiment quiz}
\label{app:numdistractors}

Before continuing to the main experiment, each participant had to correctly respond ``True'' or ``False'' to the following statements. Correct answers are given in parentheses after the statement.

\begin{itemize}
	\item The speaker can click on an object. (False)
	\item The listener wants to click on the object that the speaker is
  telling them about. (True)
  \item  The target is the object which has the red circle around it. (False)
  \item Only the speaker can send messages. (False)
  \item There are a total of 72 rounds. (True)
  \item The locations of the three objects are the same for the speaker and the listener. (False)
\end{itemize}



\red{report Gatt et al 2011 replication}

\bibliographystyle{apacite}

\setlength{\bibleftmargin}{.125in}
\setlength{\bibindent}{-\bibleftmargin}

\bibliography{bibs}


\end{document}
